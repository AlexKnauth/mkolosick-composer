\documentclass{article}

\usepackage{hyperref}


\makeatletter
\newcommand{\makeheader}{%
  \noindent
  \begin{tabular}{@{}ll}
    \ifdefined\@recipient
      \textsc{To: } & \@recipient{} \\
    \fi
    \ifdefined\@from
      \textsc{From: } & \@from{} \\
    \fi
    \ifdefined\@subject
      \textsc{Subject: } & \@subject{} \\
    \fi
  \end{tabular}
  \vspace{1\baselineskip}
}
\newcommand{\recipient}[1]{\newcommand{\@recipient}{#1}}
\newcommand{\from}[1]{\newcommand{\@from}{#1}}
\newcommand{\subject}[1]{\newcommand{\@subject}{#1}}

\renewcommand\section{\@startsection {section}{1}{\z@}%
    {\z@}
    {-1em}%
    {\itshape}}
\makeatother

\newcommand{\music}{\texttt{\#lang music}}

\setlength{\parindent}{0}
\setlength{\parskip}{10pt}

\recipient{Matthias Felleisen}
\from{Jared Gentner and Matthew Kolosick}
\subject{Design Proposal for \music}

\pagenumbering{gobble}

\begin{document}

\makeheader{}

\music{} is a language designed to enable the writing and verification of Western
classical music. The language provides a constraint sub-language for specifying
stylistic constraints on the music being written. These include constraints on
the overall form and on the harmonies.

These constraints generate a notation hashlang that enforces the stylistic
constraints on the piece being written. This language provides forms for writing
multiple voices with key and time signatures. 
\section*{Language Specification}

Our specification of grammar and scoping rules is available at
\url{https://github.ccs.neu.edu/kolosick/music/blob/master/spec.org}.

\section*{Milestones}

We will plan to follow these steps in developing \music{}.
\begin{enumerate}
  \item We will write the notation language.
  \item We will write a harmonic analyzer that handles a single key with no
    non-chord tones.
  \item We will then write the constraint sub-language that only enforces
    constraints using the previously written analyzer.
  \item We will expand the harmonic analyzer to handle key modulations.
  \item We will then write the analyzer for form constraints integrating this
    analysis into the constraint sub-language.
  \item We will then expand the harmonic analysis to allow non-chord tones.
\end{enumerate}

\end{document}